\documentclass{article}
\usepackage{graphicx} % Required for inserting images
\usepackage{float} % Required for precise figure placement
\usepackage{listings}
\usepackage{color}

\definecolor{dkgreen}{rgb}{0,0.6,0}
\definecolor{gray}{rgb}{0.5,0.5,0.5}
\definecolor{mauve}{rgb}{0.58,0,0.82}

\lstset{frame=tb,
  language=C,
  aboveskip=3mm,
  belowskip=3mm,
  showstringspaces=false,
  columns=flexible,
  basicstyle={\small\ttfamily},
  numbers=none,
  numberstyle=\tiny\color{gray},
  keywordstyle=\color{blue},
  commentstyle=\color{dkgreen},
  stringstyle=\color{mauve},
  breaklines=true,
  breakatwhitespace=true,
  tabsize=3
}

\title{Distributed System - Practical Work 1 Report \\ \large File transfer over TCP/IP in Command Line Interface (CLI)}
\author{Nguyen Hai Nam - 22BI13322}
\date{November 2024}

\begin{document}

\maketitle

\section{Introduction}
This implements a basic file transfer over TCP/IP in CLI (Command Line Interface) between a server and a client via sockets. The server waits for connections and receives a file from the client. The client sends the file in chunks, ensuring reliable transfer before both sides close their sockets.
\section{Protocol Design}
\begin{figure}[H]
    \centering
    \includegraphics[height=0.4\textheight]{Presentation1.jpg}
    \caption{File Transfer Protocol Design}
\end{figure}
The protocol for the file transfer system is designed as in the figure below. The server initializes by creating a socket and binding it to a specific address and port. The server uses \texttt{listen()}, which puts the server socket in a passive mode, where it waits for the client to approach the server to make a connection. The client initiates communication by creating a socket and connecting to the server. Once the connection is established, the client-server session begins. 
\\In this session, the client sends file data using \texttt{write()} (or \texttt{send()} in the code), and the server receives the data using \texttt{read()} (or \texttt{recv()} in the code). The server creates a file named \texttt{received\_file.txt} and writes the data into the file. This will continue until all the data is written into the file. After the session is completed, the connection is closed. Hence, we can receive a new file with the same data on the server's end.
\section{System Organization}
The provided image represents the system organization:
\begin{figure}[H]
    \centering
    \includegraphics[width=1\linewidth]{Presentation2.jpg}
    \caption{System Organization}
\end{figure}
\section{Code Snippet}
I have the following codes for both client and server:
\subsection{server.c}
\begin{lstlisting}
// server.c
#include <stdio.h>
#include <stdlib.h>
#include <string.h>
#include <arpa/inet.h>
#include <unistd.h>

#define SERVER_PORT 8080
#define BUFFER_SIZE 1024

int main() {
    int server_socket, client_socket;
    struct sockaddr_in address;
    int addrlen = sizeof(address);
    char buffer[BUFFER_SIZE] = {0};

    // Create socket
    if ((server_socket = socket(AF_INET, SOCK_STREAM, 0)) == 0) {
        perror("Socket failed");    
        exit(EXIT_FAILURE);
    }

    // Set up the server address structure
    address.sin_family = AF_INET;
    address.sin_addr.s_addr = INADDR_ANY;
    address.sin_port = htons(SERVER_PORT);

    // Bind the socket to the address and port number
    if (bind(server_socket, (struct sockaddr *)&address, sizeof(address)) < 0) {
        perror("Bind failed");
        close(server_socket);
        exit(EXIT_FAILURE);
    }

    // Listen for connections
    if (listen(server_socket, 3) < 0) {
        perror("Listen failed");
        close(server_socket);
        exit(EXIT_FAILURE);
    }
    printf("Server listening on port %d...\n", SERVER_PORT);

    // Accept a client connection
    if ((client_socket = accept(server_socket, (struct sockaddr *)&address, 
                             (socklen_t *)&addrlen)) < 0) {
        perror("Accept failed");
        close(server_socket);
        exit(EXIT_FAILURE);
    }
    printf("Client connected.\n");

    // Receive file
    FILE *file = fopen("received_file.txt", "wb");
    if (file == NULL) {
        perror("File open failed");
        close(client_socket);
        close(server_socket);
        exit(EXIT_FAILURE);
    }

    int bytes_read;
    while ((bytes_read = recv(client_socket, buffer, BUFFER_SIZE, 0)) > 0) {
        fwrite(buffer, 1, bytes_read, file);
    }
    printf("File received and saved as 'received_file.txt'\n");

    fclose(file);
    close(client_socket);
    close(server_socket);

    return 0;
}
\end{lstlisting}
\subsection{client.c}
\begin{lstlisting}
// client.c
#include <stdio.h>
#include <stdlib.h>
#include <string.h>
#include <unistd.h>
#include <arpa/inet.h>

#define SERVER_IP "127.0.0.1" 
#define SERVER_PORT 8080
#define BUFFER_SIZE 1024

int main() {
    int client_socket;
    struct sockaddr_in address;
    char buffer[BUFFER_SIZE];

    // Create a socket
    if ((client_socket = socket(AF_INET, SOCK_STREAM, 0)) < 0) {
        perror("Socket creation failed");
        exit(EXIT_FAILURE);
    }

    // Set up the server address structure
    address.sin_family = AF_INET;
    address.sin_port = htons(SERVER_PORT);

    // Convert from text to binary form
    if (inet_pton(AF_INET, SERVER_IP, &address.sin_addr) <= 0) {
        perror("inet_pton");
        close(client_socket);
        exit(EXIT_FAILURE);
    }

    // Connect to server
    if (connect(client_socket, (struct sockaddr *)&address, sizeof(address)) < 0) {
        perror("connect");
        close(client_socket);
        exit(EXIT_FAILURE);
    }
    
    printf("Connected to server %s:%d\n", SERVER_IP, SERVER_PORT);

    // Send file
    FILE *file = fopen("man.txt", "rb");
    if (file == NULL) {
        perror("File open failed");
        close(client_socket);
        exit(EXIT_FAILURE);
    }

    int bytes_read;
    while ((bytes_read = fread(buffer, 1, BUFFER_SIZE, file)) > 0) {
        send(client_socket, buffer, bytes_read, 0);
    }
    printf("File sent successfully.\n");

    fclose(file);

    // Close the socket
    close(client_socket);

    return 0;
}
\end{lstlisting}
\section{Code Implementation}
To begin with, compile and run the server.c first:
\begin{figure}[H]
    \centering
    \includegraphics[width=1\linewidth]{Picture2.1.png}
    \caption{A running server.c}
\end{figure}
After running the server.c, the server is in the wait status.Now I compile client.c and run. In this example, I have a text file man.txt to send. In the case we want to send another file, change the file name in the client.c code. We can see that the file has been sent successfully.
\begin{figure}[H]
    \centering
    \includegraphics[width=1\linewidth]{Picture1.png}
    \caption{A running client.c}
\end{figure}
On the server.c terminal, we have the result below:
\begin{figure}[H]
    \centering
    \includegraphics[width=1\linewidth]{22.png}
    \caption{File received in Server terminal}
\end{figure}
\end{document}


